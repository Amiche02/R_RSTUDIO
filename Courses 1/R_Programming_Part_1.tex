% Options for packages loaded elsewhere
\PassOptionsToPackage{unicode}{hyperref}
\PassOptionsToPackage{hyphens}{url}
%
\documentclass[
]{article}
\usepackage{amsmath,amssymb}
\usepackage{iftex}
\ifPDFTeX
  \usepackage[T1]{fontenc}
  \usepackage[utf8]{inputenc}
  \usepackage{textcomp} % provide euro and other symbols
\else % if luatex or xetex
  \usepackage{unicode-math} % this also loads fontspec
  \defaultfontfeatures{Scale=MatchLowercase}
  \defaultfontfeatures[\rmfamily]{Ligatures=TeX,Scale=1}
\fi
\usepackage{lmodern}
\ifPDFTeX\else
  % xetex/luatex font selection
\fi
% Use upquote if available, for straight quotes in verbatim environments
\IfFileExists{upquote.sty}{\usepackage{upquote}}{}
\IfFileExists{microtype.sty}{% use microtype if available
  \usepackage[]{microtype}
  \UseMicrotypeSet[protrusion]{basicmath} % disable protrusion for tt fonts
}{}
\makeatletter
\@ifundefined{KOMAClassName}{% if non-KOMA class
  \IfFileExists{parskip.sty}{%
    \usepackage{parskip}
  }{% else
    \setlength{\parindent}{0pt}
    \setlength{\parskip}{6pt plus 2pt minus 1pt}}
}{% if KOMA class
  \KOMAoptions{parskip=half}}
\makeatother
\usepackage{xcolor}
\usepackage[margin=1in]{geometry}
\usepackage{color}
\usepackage{fancyvrb}
\newcommand{\VerbBar}{|}
\newcommand{\VERB}{\Verb[commandchars=\\\{\}]}
\DefineVerbatimEnvironment{Highlighting}{Verbatim}{commandchars=\\\{\}}
% Add ',fontsize=\small' for more characters per line
\usepackage{framed}
\definecolor{shadecolor}{RGB}{248,248,248}
\newenvironment{Shaded}{\begin{snugshade}}{\end{snugshade}}
\newcommand{\AlertTok}[1]{\textcolor[rgb]{0.94,0.16,0.16}{#1}}
\newcommand{\AnnotationTok}[1]{\textcolor[rgb]{0.56,0.35,0.01}{\textbf{\textit{#1}}}}
\newcommand{\AttributeTok}[1]{\textcolor[rgb]{0.13,0.29,0.53}{#1}}
\newcommand{\BaseNTok}[1]{\textcolor[rgb]{0.00,0.00,0.81}{#1}}
\newcommand{\BuiltInTok}[1]{#1}
\newcommand{\CharTok}[1]{\textcolor[rgb]{0.31,0.60,0.02}{#1}}
\newcommand{\CommentTok}[1]{\textcolor[rgb]{0.56,0.35,0.01}{\textit{#1}}}
\newcommand{\CommentVarTok}[1]{\textcolor[rgb]{0.56,0.35,0.01}{\textbf{\textit{#1}}}}
\newcommand{\ConstantTok}[1]{\textcolor[rgb]{0.56,0.35,0.01}{#1}}
\newcommand{\ControlFlowTok}[1]{\textcolor[rgb]{0.13,0.29,0.53}{\textbf{#1}}}
\newcommand{\DataTypeTok}[1]{\textcolor[rgb]{0.13,0.29,0.53}{#1}}
\newcommand{\DecValTok}[1]{\textcolor[rgb]{0.00,0.00,0.81}{#1}}
\newcommand{\DocumentationTok}[1]{\textcolor[rgb]{0.56,0.35,0.01}{\textbf{\textit{#1}}}}
\newcommand{\ErrorTok}[1]{\textcolor[rgb]{0.64,0.00,0.00}{\textbf{#1}}}
\newcommand{\ExtensionTok}[1]{#1}
\newcommand{\FloatTok}[1]{\textcolor[rgb]{0.00,0.00,0.81}{#1}}
\newcommand{\FunctionTok}[1]{\textcolor[rgb]{0.13,0.29,0.53}{\textbf{#1}}}
\newcommand{\ImportTok}[1]{#1}
\newcommand{\InformationTok}[1]{\textcolor[rgb]{0.56,0.35,0.01}{\textbf{\textit{#1}}}}
\newcommand{\KeywordTok}[1]{\textcolor[rgb]{0.13,0.29,0.53}{\textbf{#1}}}
\newcommand{\NormalTok}[1]{#1}
\newcommand{\OperatorTok}[1]{\textcolor[rgb]{0.81,0.36,0.00}{\textbf{#1}}}
\newcommand{\OtherTok}[1]{\textcolor[rgb]{0.56,0.35,0.01}{#1}}
\newcommand{\PreprocessorTok}[1]{\textcolor[rgb]{0.56,0.35,0.01}{\textit{#1}}}
\newcommand{\RegionMarkerTok}[1]{#1}
\newcommand{\SpecialCharTok}[1]{\textcolor[rgb]{0.81,0.36,0.00}{\textbf{#1}}}
\newcommand{\SpecialStringTok}[1]{\textcolor[rgb]{0.31,0.60,0.02}{#1}}
\newcommand{\StringTok}[1]{\textcolor[rgb]{0.31,0.60,0.02}{#1}}
\newcommand{\VariableTok}[1]{\textcolor[rgb]{0.00,0.00,0.00}{#1}}
\newcommand{\VerbatimStringTok}[1]{\textcolor[rgb]{0.31,0.60,0.02}{#1}}
\newcommand{\WarningTok}[1]{\textcolor[rgb]{0.56,0.35,0.01}{\textbf{\textit{#1}}}}
\usepackage{graphicx}
\makeatletter
\def\maxwidth{\ifdim\Gin@nat@width>\linewidth\linewidth\else\Gin@nat@width\fi}
\def\maxheight{\ifdim\Gin@nat@height>\textheight\textheight\else\Gin@nat@height\fi}
\makeatother
% Scale images if necessary, so that they will not overflow the page
% margins by default, and it is still possible to overwrite the defaults
% using explicit options in \includegraphics[width, height, ...]{}
\setkeys{Gin}{width=\maxwidth,height=\maxheight,keepaspectratio}
% Set default figure placement to htbp
\makeatletter
\def\fps@figure{htbp}
\makeatother
\setlength{\emergencystretch}{3em} % prevent overfull lines
\providecommand{\tightlist}{%
  \setlength{\itemsep}{0pt}\setlength{\parskip}{0pt}}
\setcounter{secnumdepth}{-\maxdimen} % remove section numbering
\ifLuaTeX
  \usepackage{selnolig}  % disable illegal ligatures
\fi
\IfFileExists{bookmark.sty}{\usepackage{bookmark}}{\usepackage{hyperref}}
\IfFileExists{xurl.sty}{\usepackage{xurl}}{} % add URL line breaks if available
\urlstyle{same}
\hypersetup{
  pdftitle={R Programming, Part 1},
  pdfauthor={Feryal WINDAL},
  hidelinks,
  pdfcreator={LaTeX via pandoc}}

\title{R Programming, Part 1}
\author{Feryal WINDAL}
\date{4/22/2021}

\begin{document}
\maketitle

In this first part of the course, we will understand the R language.
First, we will explore the types of data that we can meet such as real
numbers, characters, booleans\ldots and understand the notion of
variable. Secondly, we will work on data structures, arrays (vectors)
and matrices with the corresponding instructions.

\hypertarget{data-class-and-type}{%
\section{Data class and type}\label{data-class-and-type}}

To begin, we will talk about the numeric data class that represents
numbers and digits in R. \#\# Numerics

\begin{Shaded}
\begin{Highlighting}[]
\FunctionTok{class}\NormalTok{(}\FloatTok{6.5}\NormalTok{)}
\end{Highlighting}
\end{Shaded}

\begin{verbatim}
## [1] "numeric"
\end{verbatim}

\begin{Shaded}
\begin{Highlighting}[]
\FunctionTok{class}\NormalTok{(}\DecValTok{6}\NormalTok{)}
\end{Highlighting}
\end{Shaded}

\begin{verbatim}
## [1] "numeric"
\end{verbatim}

\begin{Shaded}
\begin{Highlighting}[]
\FunctionTok{typeof}\NormalTok{(}\FloatTok{6.5}\NormalTok{)}
\end{Highlighting}
\end{Shaded}

\begin{verbatim}
## [1] "double"
\end{verbatim}

\begin{Shaded}
\begin{Highlighting}[]
\FunctionTok{as.integer}\NormalTok{(}\FloatTok{6.5}\NormalTok{)}
\end{Highlighting}
\end{Shaded}

\begin{verbatim}
## [1] 6
\end{verbatim}

\begin{Shaded}
\begin{Highlighting}[]
\FunctionTok{typeof}\NormalTok{(}\FunctionTok{as.integer}\NormalTok{(}\DecValTok{4}\NormalTok{))}
\end{Highlighting}
\end{Shaded}

\begin{verbatim}
## [1] "integer"
\end{verbatim}

\begin{Shaded}
\begin{Highlighting}[]
\CommentTok{\#typeof(as.integer(6.5))}
\end{Highlighting}
\end{Shaded}

typeof (6.5) informs us that they are doubles. If we want to transform
this number into an int then we have to use as.integer (6.5) which
returns us 6. working with integers or integrating does not take up much
space in your machine's memory. However, in this course, we will use the
doubles.

\hypertarget{characters}{%
\subsection{Characters}\label{characters}}

\begin{Shaded}
\begin{Highlighting}[]
\FunctionTok{class}\NormalTok{(}\StringTok{"hello"}\NormalTok{)}
\end{Highlighting}
\end{Shaded}

\begin{verbatim}
## [1] "character"
\end{verbatim}

\begin{Shaded}
\begin{Highlighting}[]
\FunctionTok{typeof}\NormalTok{(}\StringTok{"hello"}\NormalTok{)}
\end{Highlighting}
\end{Shaded}

\begin{verbatim}
## [1] "character"
\end{verbatim}

\begin{Shaded}
\begin{Highlighting}[]
\FunctionTok{typeof}\NormalTok{(}\StringTok{"v"}\NormalTok{)}
\end{Highlighting}
\end{Shaded}

\begin{verbatim}
## [1] "character"
\end{verbatim}

\hypertarget{booleans}{%
\subsection{Booleans}\label{booleans}}

Another class that we will discuss later is the Boolean class also
called logical in R. This is true or false for example

\begin{Shaded}
\begin{Highlighting}[]
\DecValTok{44}\SpecialCharTok{\textgreater{}}\DecValTok{23}
\end{Highlighting}
\end{Shaded}

\begin{verbatim}
## [1] TRUE
\end{verbatim}

\begin{Shaded}
\begin{Highlighting}[]
\DecValTok{44}\SpecialCharTok{\textless{}}\DecValTok{23}
\end{Highlighting}
\end{Shaded}

\begin{verbatim}
## [1] FALSE
\end{verbatim}

\hypertarget{variables}{%
\subsection{Variables}\label{variables}}

A variable will allow us to store information. For example, when we need
to get a file containing data, we will store it in a variable. So we can
manipulate all of this data just by using a variable.

A variable will take several types of data as we saw earlier.

\begin{Shaded}
\begin{Highlighting}[]
\NormalTok{my\_variable }\OtherTok{=} \DecValTok{4}
\NormalTok{my\_variable}
\end{Highlighting}
\end{Shaded}

\begin{verbatim}
## [1] 4
\end{verbatim}

\begin{Shaded}
\begin{Highlighting}[]
\FunctionTok{typeof}\NormalTok{(my\_variable)}
\end{Highlighting}
\end{Shaded}

\begin{verbatim}
## [1] "double"
\end{verbatim}

\begin{Shaded}
\begin{Highlighting}[]
\FunctionTok{class}\NormalTok{(my\_variable)}
\end{Highlighting}
\end{Shaded}

\begin{verbatim}
## [1] "numeric"
\end{verbatim}

\begin{Shaded}
\begin{Highlighting}[]
\NormalTok{my\_caractere }\OtherTok{=}\StringTok{"hello"}
\NormalTok{my\_caractere}
\end{Highlighting}
\end{Shaded}

\begin{verbatim}
## [1] "hello"
\end{verbatim}

\begin{Shaded}
\begin{Highlighting}[]
\FunctionTok{typeof}\NormalTok{(my\_caractere)}
\end{Highlighting}
\end{Shaded}

\begin{verbatim}
## [1] "character"
\end{verbatim}

\begin{Shaded}
\begin{Highlighting}[]
\FunctionTok{class}\NormalTok{(my\_caractere)}
\end{Highlighting}
\end{Shaded}

\begin{verbatim}
## [1] "character"
\end{verbatim}

\begin{Shaded}
\begin{Highlighting}[]
\NormalTok{my\_booleen}\OtherTok{=} \ConstantTok{TRUE}

\NormalTok{my\_booleen}
\end{Highlighting}
\end{Shaded}

\begin{verbatim}
## [1] TRUE
\end{verbatim}

\begin{Shaded}
\begin{Highlighting}[]
\FunctionTok{typeof}\NormalTok{(my\_booleen)}
\end{Highlighting}
\end{Shaded}

\begin{verbatim}
## [1] "logical"
\end{verbatim}

\begin{Shaded}
\begin{Highlighting}[]
\NormalTok{bool}\OtherTok{=} \DecValTok{4}\SpecialCharTok{\textless{}}\DecValTok{2}
\NormalTok{bool}
\end{Highlighting}
\end{Shaded}

\begin{verbatim}
## [1] FALSE
\end{verbatim}

\begin{Shaded}
\begin{Highlighting}[]
\NormalTok{Booleen }\OtherTok{=} \ConstantTok{TRUE}
\end{Highlighting}
\end{Shaded}

\hypertarget{arithmetic-operations}{%
\section{Arithmetic operations}\label{arithmetic-operations}}

What you need to know is that R works like a calculator

\begin{Shaded}
\begin{Highlighting}[]
\DecValTok{4}\SpecialCharTok{+}\DecValTok{6}
\end{Highlighting}
\end{Shaded}

\begin{verbatim}
## [1] 10
\end{verbatim}

\begin{Shaded}
\begin{Highlighting}[]
\NormalTok{var1 }\OtherTok{=} \DecValTok{100} \DocumentationTok{\#\# new variable named var1 which takes the value 100}
\NormalTok{var2 }\OtherTok{=} \DecValTok{4} \DocumentationTok{\#\# var2 takes the value 4}
\NormalTok{var1 }\SpecialCharTok{+}\NormalTok{ var2 }\DocumentationTok{\#\# addition between variables}
\end{Highlighting}
\end{Shaded}

\begin{verbatim}
## [1] 104
\end{verbatim}

\begin{Shaded}
\begin{Highlighting}[]
\NormalTok{var3 }\OtherTok{=}\NormalTok{ var1 }\SpecialCharTok{+}\NormalTok{ var2 }\CommentTok{\# assign the result of the addition to a new variable var3}
\NormalTok{var3 }\DocumentationTok{\#\# display of var3}
\end{Highlighting}
\end{Shaded}

\begin{verbatim}
## [1] 104
\end{verbatim}

\begin{Shaded}
\begin{Highlighting}[]
\NormalTok{var1 }\SpecialCharTok{*}\NormalTok{ var2 }\DocumentationTok{\#\# multiplication}
\end{Highlighting}
\end{Shaded}

\begin{verbatim}
## [1] 400
\end{verbatim}

\begin{Shaded}
\begin{Highlighting}[]
\NormalTok{var1 }\SpecialCharTok{/}\NormalTok{ var2 }\DocumentationTok{\#\# division}
\end{Highlighting}
\end{Shaded}

\begin{verbatim}
## [1] 25
\end{verbatim}

\begin{Shaded}
\begin{Highlighting}[]
\NormalTok{var1 }\SpecialCharTok{\%\%}\NormalTok{ var2 }\DocumentationTok{\#\# modulo operation (remainder of the division between var1 and var2)}
\end{Highlighting}
\end{Shaded}

\begin{verbatim}
## [1] 0
\end{verbatim}

\hypertarget{simple-data-structure}{%
\section{Simple data structure}\label{simple-data-structure}}

\hypertarget{vectors}{%
\subsection{Vectors}\label{vectors}}

A vector in R has the same definition as in linear algebra. That is to
say that it is a variable representing one or more variables. If you are
familiar with other programming languages, it's like an array of values.
The vector is represented by a variable where all the values are stored.
Earlier, we declared a variable ``var1''. This also represents a vector
in R.

To create a vector we will use the function c for concatenation

\begin{Shaded}
\begin{Highlighting}[]
\NormalTok{vec1}\OtherTok{=} \FunctionTok{c}\NormalTok{(}\DecValTok{1}\NormalTok{,}\DecValTok{2}\NormalTok{,}\DecValTok{3}\NormalTok{)}
\NormalTok{vec1}
\end{Highlighting}
\end{Shaded}

\begin{verbatim}
## [1] 1 2 3
\end{verbatim}

\begin{Shaded}
\begin{Highlighting}[]
\FunctionTok{typeof}\NormalTok{(vec1)}
\end{Highlighting}
\end{Shaded}

\begin{verbatim}
## [1] "double"
\end{verbatim}

\begin{Shaded}
\begin{Highlighting}[]
\FunctionTok{class}\NormalTok{(vec1)}
\end{Highlighting}
\end{Shaded}

\begin{verbatim}
## [1] "numeric"
\end{verbatim}

If we want a character type vector

\begin{Shaded}
\begin{Highlighting}[]
\FunctionTok{c}\NormalTok{(}\StringTok{"a"}\NormalTok{,}\StringTok{"b"}\NormalTok{,}\StringTok{"c"}\NormalTok{)}
\end{Highlighting}
\end{Shaded}

\begin{verbatim}
## [1] "a" "b" "c"
\end{verbatim}

\begin{Shaded}
\begin{Highlighting}[]
\NormalTok{car1}\OtherTok{=}\FunctionTok{c}\NormalTok{(}\StringTok{"a"}\NormalTok{,}\StringTok{"b"}\NormalTok{,}\StringTok{"c"}\NormalTok{)}
\NormalTok{car1}
\end{Highlighting}
\end{Shaded}

\begin{verbatim}
## [1] "a" "b" "c"
\end{verbatim}

You cannot mix data types in a vector. The vector, like the variable,
meets the typing criteria seen previously. If we want to mix different
types, for example, well let's experiment:

\begin{Shaded}
\begin{Highlighting}[]
\NormalTok{vec2}\OtherTok{=}\FunctionTok{c}\NormalTok{(}\DecValTok{3}\NormalTok{,}\StringTok{"a"}\NormalTok{,}\FloatTok{4.5}\NormalTok{)}
\NormalTok{vec2}
\end{Highlighting}
\end{Shaded}

\begin{verbatim}
## [1] "3"   "a"   "4.5"
\end{verbatim}

\begin{Shaded}
\begin{Highlighting}[]
\FunctionTok{typeof}\NormalTok{((vec2))}
\end{Highlighting}
\end{Shaded}

\begin{verbatim}
## [1] "character"
\end{verbatim}

We note that by default, it is the type ``character'' which takes
defined.

\hypertarget{operations-on-vectors}{%
\subsection{Operations on vectors:}\label{operations-on-vectors}}

In this part, we are interested in arithmetic and other operations on
vectors. Other functions allow to generate vectors for example

\begin{Shaded}
\begin{Highlighting}[]
\CommentTok{\# with the seq function, two ways to write}
\NormalTok{my\_vector1}\OtherTok{=} \FunctionTok{seq}\NormalTok{(}\AttributeTok{from=}\DecValTok{1}\NormalTok{, }\AttributeTok{to=}\DecValTok{10}\NormalTok{)}
\NormalTok{my\_vector1}
\end{Highlighting}
\end{Shaded}

\begin{verbatim}
##  [1]  1  2  3  4  5  6  7  8  9 10
\end{verbatim}

\begin{Shaded}
\begin{Highlighting}[]
\NormalTok{my\_vector2}\OtherTok{=}\FunctionTok{seq}\NormalTok{(}\DecValTok{1}\SpecialCharTok{:}\DecValTok{10}\NormalTok{)}
\NormalTok{my\_vector2}
\end{Highlighting}
\end{Shaded}

\begin{verbatim}
##  [1]  1  2  3  4  5  6  7  8  9 10
\end{verbatim}

\begin{Shaded}
\begin{Highlighting}[]
\CommentTok{\# with the rep function, two ways to write}
\NormalTok{my\_vector3}\OtherTok{=} \FunctionTok{rep}\NormalTok{(}\DecValTok{10}\NormalTok{, }\AttributeTok{times =} \DecValTok{10}\NormalTok{)}
\NormalTok{my\_vector3}
\end{Highlighting}
\end{Shaded}

\begin{verbatim}
##  [1] 10 10 10 10 10 10 10 10 10 10
\end{verbatim}

\begin{Shaded}
\begin{Highlighting}[]
\NormalTok{my\_vector4 }\OtherTok{=} \FunctionTok{rep}\NormalTok{(}\DecValTok{10}\NormalTok{,}\DecValTok{10}\NormalTok{)}
\NormalTok{my\_vector4}
\end{Highlighting}
\end{Shaded}

\begin{verbatim}
##  [1] 10 10 10 10 10 10 10 10 10 10
\end{verbatim}

To increase the values of the vector my\_vector1 by 1

\begin{Shaded}
\begin{Highlighting}[]
\NormalTok{my\_vector1}
\end{Highlighting}
\end{Shaded}

\begin{verbatim}
##  [1]  1  2  3  4  5  6  7  8  9 10
\end{verbatim}

\begin{Shaded}
\begin{Highlighting}[]
\NormalTok{my\_vector1}\SpecialCharTok{+}\DecValTok{1}
\end{Highlighting}
\end{Shaded}

\begin{verbatim}
##  [1]  2  3  4  5  6  7  8  9 10 11
\end{verbatim}

\begin{Shaded}
\begin{Highlighting}[]
\NormalTok{my\_vector1}
\end{Highlighting}
\end{Shaded}

\begin{verbatim}
##  [1]  1  2  3  4  5  6  7  8  9 10
\end{verbatim}

To multiply the values of the vector my\_vector1 by 2

\begin{Shaded}
\begin{Highlighting}[]
\NormalTok{my\_vector1}
\end{Highlighting}
\end{Shaded}

\begin{verbatim}
##  [1]  1  2  3  4  5  6  7  8  9 10
\end{verbatim}

\begin{Shaded}
\begin{Highlighting}[]
\NormalTok{my\_vector1}\SpecialCharTok{*}\DecValTok{2}
\end{Highlighting}
\end{Shaded}

\begin{verbatim}
##  [1]  2  4  6  8 10 12 14 16 18 20
\end{verbatim}

to do multiplication operations between two vectors

\begin{Shaded}
\begin{Highlighting}[]
\NormalTok{my\_vector1}
\end{Highlighting}
\end{Shaded}

\begin{verbatim}
##  [1]  1  2  3  4  5  6  7  8  9 10
\end{verbatim}

\begin{Shaded}
\begin{Highlighting}[]
\NormalTok{my\_vector2 }\OtherTok{=} \FunctionTok{seq}\NormalTok{(}\AttributeTok{from=}\DecValTok{6}\NormalTok{, }\AttributeTok{to=} \DecValTok{15}\NormalTok{)}
\NormalTok{my\_vector2}
\end{Highlighting}
\end{Shaded}

\begin{verbatim}
##  [1]  6  7  8  9 10 11 12 13 14 15
\end{verbatim}

\begin{Shaded}
\begin{Highlighting}[]
\NormalTok{my\_vector1}\SpecialCharTok{*}\NormalTok{my\_vector2}
\end{Highlighting}
\end{Shaded}

\begin{verbatim}
##  [1]   6  14  24  36  50  66  84 104 126 150
\end{verbatim}

Same for performing additions between vectors

\begin{Shaded}
\begin{Highlighting}[]
\NormalTok{my\_vector1}
\end{Highlighting}
\end{Shaded}

\begin{verbatim}
##  [1]  1  2  3  4  5  6  7  8  9 10
\end{verbatim}

\begin{Shaded}
\begin{Highlighting}[]
\NormalTok{my\_vector2}
\end{Highlighting}
\end{Shaded}

\begin{verbatim}
##  [1]  6  7  8  9 10 11 12 13 14 15
\end{verbatim}

\begin{Shaded}
\begin{Highlighting}[]
\NormalTok{my\_vector1}\SpecialCharTok{+}\NormalTok{my\_vector2}
\end{Highlighting}
\end{Shaded}

\begin{verbatim}
##  [1]  7  9 11 13 15 17 19 21 23 25
\end{verbatim}

To perform a subtraction

\begin{Shaded}
\begin{Highlighting}[]
\NormalTok{my\_vector1}
\end{Highlighting}
\end{Shaded}

\begin{verbatim}
##  [1]  1  2  3  4  5  6  7  8  9 10
\end{verbatim}

\begin{Shaded}
\begin{Highlighting}[]
\NormalTok{my\_vector2}
\end{Highlighting}
\end{Shaded}

\begin{verbatim}
##  [1]  6  7  8  9 10 11 12 13 14 15
\end{verbatim}

\begin{Shaded}
\begin{Highlighting}[]
\NormalTok{my\_vector1 }\SpecialCharTok{{-}}\NormalTok{ my\_vector2}
\end{Highlighting}
\end{Shaded}

\begin{verbatim}
##  [1] -5 -5 -5 -5 -5 -5 -5 -5 -5 -5
\end{verbatim}

To divide between vectors

\begin{Shaded}
\begin{Highlighting}[]
\NormalTok{my\_vector1}
\end{Highlighting}
\end{Shaded}

\begin{verbatim}
##  [1]  1  2  3  4  5  6  7  8  9 10
\end{verbatim}

\begin{Shaded}
\begin{Highlighting}[]
\NormalTok{my\_vector2}
\end{Highlighting}
\end{Shaded}

\begin{verbatim}
##  [1]  6  7  8  9 10 11 12 13 14 15
\end{verbatim}

\begin{Shaded}
\begin{Highlighting}[]
\NormalTok{my\_vector1 }\SpecialCharTok{/}\NormalTok{ my\_vector2}
\end{Highlighting}
\end{Shaded}

\begin{verbatim}
##  [1] 0.1666667 0.2857143 0.3750000 0.4444444 0.5000000 0.5454545 0.5833333
##  [8] 0.6153846 0.6428571 0.6666667
\end{verbatim}

To concatenate two vectors, just do the following operation

\begin{Shaded}
\begin{Highlighting}[]
\NormalTok{my\_vector1}
\end{Highlighting}
\end{Shaded}

\begin{verbatim}
##  [1]  1  2  3  4  5  6  7  8  9 10
\end{verbatim}

\begin{Shaded}
\begin{Highlighting}[]
\NormalTok{my\_vector3}
\end{Highlighting}
\end{Shaded}

\begin{verbatim}
##  [1] 10 10 10 10 10 10 10 10 10 10
\end{verbatim}

\begin{Shaded}
\begin{Highlighting}[]
\NormalTok{my\_vector5}\OtherTok{=}\FunctionTok{c}\NormalTok{(my\_vector1,my\_vector3)}
\NormalTok{my\_vector5}
\end{Highlighting}
\end{Shaded}

\begin{verbatim}
##  [1]  1  2  3  4  5  6  7  8  9 10 10 10 10 10 10 10 10 10 10 10
\end{verbatim}

\hypertarget{the-different-functions-for-manipulating-vectors}{%
\subsection{The different functions for manipulating
vectors:}\label{the-different-functions-for-manipulating-vectors}}

In this section we will focus on existing and very useful functions in R
which will allow us to manipulate vectors:

\begin{Shaded}
\begin{Highlighting}[]
\FunctionTok{length}\NormalTok{(my\_vector1)  }\DocumentationTok{\#\# vector size}
\end{Highlighting}
\end{Shaded}

\begin{verbatim}
## [1] 10
\end{verbatim}

The length function takes a vector as argument.If we want for example to
sort a vector from the smallest value to the largest value:

\begin{Shaded}
\begin{Highlighting}[]
\NormalTok{my\_vector6 }\OtherTok{=} \FunctionTok{c}\NormalTok{(}\DecValTok{49}\NormalTok{, }\DecValTok{24}\NormalTok{, }\DecValTok{2}\NormalTok{, }\DecValTok{87}\NormalTok{, }\DecValTok{6}\NormalTok{, }\DecValTok{10}\NormalTok{, }\DecValTok{33}\NormalTok{, }\DecValTok{56}\NormalTok{, }\DecValTok{88}\NormalTok{, }\DecValTok{17}\NormalTok{) }\DocumentationTok{\#\# creation of a new vector}
\FunctionTok{sort}\NormalTok{(my\_vector6) }\DocumentationTok{\#\# the sort function allows you to sort the values inside a vector.}
\end{Highlighting}
\end{Shaded}

\begin{verbatim}
##  [1]  2  6 10 17 24 33 49 56 87 88
\end{verbatim}

If we want to create a new vector with the sorted values of the
my\_vector6 vector then

\begin{Shaded}
\begin{Highlighting}[]
\NormalTok{mon\_vecteurTrieC }\OtherTok{=} \FunctionTok{sort}\NormalTok{(my\_vector6)}
\NormalTok{mon\_vecteurTrieC}
\end{Highlighting}
\end{Shaded}

\begin{verbatim}
##  [1]  2  6 10 17 24 33 49 56 87 88
\end{verbatim}

Now if we want to sort descending, we just need to use the same output
function with an option which is decreasing = TRUE:

\begin{Shaded}
\begin{Highlighting}[]
\FunctionTok{sort}\NormalTok{(my\_vector6, }\AttributeTok{decreasing=} \ConstantTok{TRUE}\NormalTok{)}
\end{Highlighting}
\end{Shaded}

\begin{verbatim}
##  [1] 88 87 56 49 33 24 17 10  6  2
\end{verbatim}

If we can also create a new variable or object to which we can assign
this vector sorted in descending order:

\begin{Shaded}
\begin{Highlighting}[]
\NormalTok{mon\_vecteurTrieD}\OtherTok{=}\FunctionTok{sort}\NormalTok{(my\_vector6, }\AttributeTok{decreasing =} \ConstantTok{TRUE}\NormalTok{)}
\NormalTok{mon\_vecteurTrieD}
\end{Highlighting}
\end{Shaded}

\begin{verbatim}
##  [1] 88 87 56 49 33 24 17 10  6  2
\end{verbatim}

To see the ranks of each value in my array, we can use the rank
function:

\begin{Shaded}
\begin{Highlighting}[]
\NormalTok{my\_vector6}
\end{Highlighting}
\end{Shaded}

\begin{verbatim}
##  [1] 49 24  2 87  6 10 33 56 88 17
\end{verbatim}

\begin{Shaded}
\begin{Highlighting}[]
\FunctionTok{rank}\NormalTok{(my\_vector6) }
\end{Highlighting}
\end{Shaded}

\begin{verbatim}
##  [1]  7  5  1  9  2  3  6  8 10  4
\end{verbatim}

\begin{Shaded}
\begin{Highlighting}[]
\FunctionTok{rank}\NormalTok{(mon\_vecteurTrieC)}
\end{Highlighting}
\end{Shaded}

\begin{verbatim}
##  [1]  1  2  3  4  5  6  7  8  9 10
\end{verbatim}

\begin{Shaded}
\begin{Highlighting}[]
\FunctionTok{rank}\NormalTok{(mon\_vecteurTrieD)}
\end{Highlighting}
\end{Shaded}

\begin{verbatim}
##  [1] 10  9  8  7  6  5  4  3  2  1
\end{verbatim}

\hypertarget{browse-vectors}{%
\subsection{Browse vectors}\label{browse-vectors}}

If, for example, we want to know the value at position 2 of the vector
my\_vector6

\begin{Shaded}
\begin{Highlighting}[]
\NormalTok{my\_vector6}
\end{Highlighting}
\end{Shaded}

\begin{verbatim}
##  [1] 49 24  2 87  6 10 33 56 88 17
\end{verbatim}

\begin{Shaded}
\begin{Highlighting}[]
\NormalTok{my\_vector6[}\DecValTok{2}\NormalTok{]}
\end{Highlighting}
\end{Shaded}

\begin{verbatim}
## [1] 24
\end{verbatim}

If we now want to display the first 3 values of the vector:

\begin{Shaded}
\begin{Highlighting}[]
\NormalTok{my\_vector6 [}\DecValTok{1}\SpecialCharTok{:} \DecValTok{3}\NormalTok{] }\DocumentationTok{\#\# the 1: 3 sequence indicates that I want the values from position 1 to position 3}
\end{Highlighting}
\end{Shaded}

\begin{verbatim}
## [1] 49 24  2
\end{verbatim}

\begin{Shaded}
\begin{Highlighting}[]
\NormalTok{my\_vector6 [}\DecValTok{3}\SpecialCharTok{:} \DecValTok{7}\NormalTok{] }\DocumentationTok{\#\# displays values from position 3 to position 7}
\end{Highlighting}
\end{Shaded}

\begin{verbatim}
## [1]  2 87  6 10 33
\end{verbatim}

If we now wish to access positions 3, 5, 9 of the vector

\begin{Shaded}
\begin{Highlighting}[]
\NormalTok{my\_vector6}
\end{Highlighting}
\end{Shaded}

\begin{verbatim}
##  [1] 49 24  2 87  6 10 33 56 88 17
\end{verbatim}

\begin{Shaded}
\begin{Highlighting}[]
\NormalTok{my\_vector6[}\DecValTok{3}\NormalTok{]}
\end{Highlighting}
\end{Shaded}

\begin{verbatim}
## [1] 2
\end{verbatim}

\begin{Shaded}
\begin{Highlighting}[]
\NormalTok{my\_vector6[}\FunctionTok{c}\NormalTok{(}\DecValTok{3}\NormalTok{,}\DecValTok{5}\NormalTok{,}\DecValTok{9}\NormalTok{)]}
\end{Highlighting}
\end{Shaded}

\begin{verbatim}
## [1]  2  6 88
\end{verbatim}

If you want to retrieve all the values greater than 30

\begin{Shaded}
\begin{Highlighting}[]
\NormalTok{my\_vector6}
\end{Highlighting}
\end{Shaded}

\begin{verbatim}
##  [1] 49 24  2 87  6 10 33 56 88 17
\end{verbatim}

\begin{Shaded}
\begin{Highlighting}[]
\NormalTok{superior }\OtherTok{=}\NormalTok{ my\_vector6}\SpecialCharTok{\textgreater{}} \DecValTok{30} \DocumentationTok{\#\# creation of a new vector}
\NormalTok{superior }\DocumentationTok{\#\# here we can see that when it is a value greater than 30, it displays TRUE and when it is a value less than 30, it displays FALSE}
\end{Highlighting}
\end{Shaded}

\begin{verbatim}
##  [1]  TRUE FALSE FALSE  TRUE FALSE FALSE  TRUE  TRUE  TRUE FALSE
\end{verbatim}

\begin{Shaded}
\begin{Highlighting}[]
\NormalTok{my\_vector6[superior] }\DocumentationTok{\#\# now, showing all values greater than 30}
\end{Highlighting}
\end{Shaded}

\begin{verbatim}
## [1] 49 87 33 56 88
\end{verbatim}

Ici, nous allons nous interesser à certaines fonctions qui permettent de
faire certaines opérations sur les vecteurs:

\begin{Shaded}
\begin{Highlighting}[]
\NormalTok{my\_vector6}
\end{Highlighting}
\end{Shaded}

\begin{verbatim}
##  [1] 49 24  2 87  6 10 33 56 88 17
\end{verbatim}

\begin{Shaded}
\begin{Highlighting}[]
\FunctionTok{sum}\NormalTok{ (my\_vector6) }\CommentTok{\# sum of all vector values}
\end{Highlighting}
\end{Shaded}

\begin{verbatim}
## [1] 372
\end{verbatim}

\begin{Shaded}
\begin{Highlighting}[]
\FunctionTok{mean}\NormalTok{ (my\_vector6) }\CommentTok{\# mean of vector}
\end{Highlighting}
\end{Shaded}

\begin{verbatim}
## [1] 37.2
\end{verbatim}

\begin{Shaded}
\begin{Highlighting}[]
\FunctionTok{min}\NormalTok{ (my\_vector6) }\CommentTok{\# know the smallest value of the vector}
\end{Highlighting}
\end{Shaded}

\begin{verbatim}
## [1] 2
\end{verbatim}

\begin{Shaded}
\begin{Highlighting}[]
\FunctionTok{max}\NormalTok{ (my\_vector6) }\CommentTok{\# know the greatest value of the vector}
\end{Highlighting}
\end{Shaded}

\begin{verbatim}
## [1] 88
\end{verbatim}

\begin{Shaded}
\begin{Highlighting}[]
\FunctionTok{median}\NormalTok{ (my\_vector6)}
\end{Highlighting}
\end{Shaded}

\begin{verbatim}
## [1] 28.5
\end{verbatim}

\begin{Shaded}
\begin{Highlighting}[]
\FunctionTok{summary}\NormalTok{ (my\_vector6) }\CommentTok{\# informs us of vector distributions}
\end{Highlighting}
\end{Shaded}

\begin{verbatim}
##    Min. 1st Qu.  Median    Mean 3rd Qu.    Max. 
##    2.00   11.75   28.50   37.20   54.25   88.00
\end{verbatim}

\hypertarget{simple-statistics-on-a-vector}{%
\subsection{Simple statistics on a
vector}\label{simple-statistics-on-a-vector}}

In this part, we are going to review the set of functions that we have
seen previously by illustrating this with a simple exercise that speaks
to everyone, that is to say the notes in a class. Suppose we have math
scores for a class of 25 students and we want to know the different
statistics. In this case, it means knowing the average, the worst score,
the highest score, those who are below the average, those who are above
the average, etc.

For this, we will use a runif function, which will generate values
randomly between a min and a max. Then we want to round these values so
that they are more integer type.

\begin{Shaded}
\begin{Highlighting}[]
\NormalTok{grade\_math }\OtherTok{=} \FunctionTok{runif}\NormalTok{ (}\DecValTok{25}\NormalTok{, }\AttributeTok{min =} \DecValTok{1}\NormalTok{, }\AttributeTok{max =} \DecValTok{20}\NormalTok{) }\CommentTok{\# randomly generate 25 values between 1 and 20}
\NormalTok{grade\_math }\CommentTok{\# in this vector we can see that these are random values with lots of decimal places.}
\end{Highlighting}
\end{Shaded}

\begin{verbatim}
##  [1]  3.300092  2.396880 18.746314 18.792320 18.992879 17.805648 18.345717
##  [8] 16.217433 15.230759  8.221450 19.352025  8.229528  1.213513  8.115575
## [15] 17.173778 16.873690  1.520676 19.914897 12.565446  1.631267 12.752673
## [22]  7.953478  4.871654  7.045251 16.610009
\end{verbatim}

\begin{Shaded}
\begin{Highlighting}[]
\NormalTok{grade\_math }\OtherTok{=} \FunctionTok{round}\NormalTok{(grade\_math, }\AttributeTok{digits =} \DecValTok{0}\NormalTok{) }\CommentTok{\# here we are going to round the numbers generated by runif to any number after the decimal point with the round function.}
\NormalTok{grade\_math}
\end{Highlighting}
\end{Shaded}

\begin{verbatim}
##  [1]  3  2 19 19 19 18 18 16 15  8 19  8  1  8 17 17  2 20 13  2 13  8  5  7 17
\end{verbatim}

Now we are going to give names to the students

\begin{Shaded}
\begin{Highlighting}[]
\FunctionTok{names}\NormalTok{(grade\_math) }\OtherTok{=} \FunctionTok{c}\NormalTok{(}\StringTok{"Elise"}\NormalTok{, }\StringTok{"Leon"}\NormalTok{,}\StringTok{"Sully"}\NormalTok{, }\StringTok{"Anouck"}\NormalTok{,}\StringTok{"Baptiste"}\NormalTok{, }\StringTok{"Thibault"}\NormalTok{, }\StringTok{"Paul"}\NormalTok{, }\StringTok{"Zoé"}\NormalTok{, }\StringTok{"Zély"}\NormalTok{, }\StringTok{"Amandine"}\NormalTok{, }\StringTok{"Antoine"}\NormalTok{, }\StringTok{"Charles"}\NormalTok{, }\StringTok{"Lilou"}\NormalTok{, }\StringTok{"Chloé"}\NormalTok{, }\StringTok{"Daphnée"}\NormalTok{, }\StringTok{"Nicolas"}\NormalTok{, }\StringTok{"Maxcence"}\NormalTok{, }\StringTok{"Auguste"}\NormalTok{, }\StringTok{"Mathurin"}\NormalTok{, }\StringTok{"Sarah"}\NormalTok{, }\StringTok{"Lili"}\NormalTok{, }\StringTok{"Marion"}\NormalTok{,}\StringTok{"Naomie"}\NormalTok{,}\StringTok{"Raphael"}\NormalTok{, }\StringTok{"Flora"}\NormalTok{)}
\NormalTok{grade\_math}
\end{Highlighting}
\end{Shaded}

\begin{verbatim}
##    Elise     Leon    Sully   Anouck Baptiste Thibault     Paul      Zoé 
##        3        2       19       19       19       18       18       16 
##     Zély Amandine  Antoine  Charles    Lilou    Chloé  Daphnée  Nicolas 
##       15        8       19        8        1        8       17       17 
## Maxcence  Auguste Mathurin    Sarah     Lili   Marion   Naomie  Raphael 
##        2       20       13        2       13        8        5        7 
##    Flora 
##       17
\end{verbatim}

\begin{Shaded}
\begin{Highlighting}[]
\FunctionTok{typeof}\NormalTok{(grade\_math)}
\end{Highlighting}
\end{Shaded}

\begin{verbatim}
## [1] "double"
\end{verbatim}

\begin{Shaded}
\begin{Highlighting}[]
\FunctionTok{class}\NormalTok{(grade\_math)}
\end{Highlighting}
\end{Shaded}

\begin{verbatim}
## [1] "numeric"
\end{verbatim}

\hypertarget{edit-vector-values}{%
\subsection{Edit vector values}\label{edit-vector-values}}

\begin{Shaded}
\begin{Highlighting}[]
\NormalTok{grade\_math[}\DecValTok{3}\NormalTok{]}\OtherTok{=} \DecValTok{14}
\NormalTok{grade\_math[}\StringTok{"Baptiste"}\NormalTok{]}\OtherTok{=} \DecValTok{13}
\FunctionTok{names}\NormalTok{(grade\_math[}\DecValTok{2}\NormalTok{])}\OtherTok{=}\FunctionTok{c}\NormalTok{(}\StringTok{"Mikael"}\NormalTok{)}
\NormalTok{grade\_math}
\end{Highlighting}
\end{Shaded}

\begin{verbatim}
##    Elise     Leon    Sully   Anouck Baptiste Thibault     Paul      Zoé 
##        3        2       14       19       13       18       18       16 
##     Zély Amandine  Antoine  Charles    Lilou    Chloé  Daphnée  Nicolas 
##       15        8       19        8        1        8       17       17 
## Maxcence  Auguste Mathurin    Sarah     Lili   Marion   Naomie  Raphael 
##        2       20       13        2       13        8        5        7 
##    Flora 
##       17
\end{verbatim}

\hypertarget{matrices}{%
\section{Matrices}\label{matrices}}

It is a table with two dimensions.

\begin{Shaded}
\begin{Highlighting}[]
\NormalTok{vector1}\OtherTok{=} \FunctionTok{sample}\NormalTok{(}\DecValTok{1}\SpecialCharTok{:}\DecValTok{20}\NormalTok{, }\DecValTok{12}\NormalTok{)}
\NormalTok{vector1}
\end{Highlighting}
\end{Shaded}

\begin{verbatim}
##  [1]  4 14  7 18 10  5 12  9  1 13 17  2
\end{verbatim}

\begin{Shaded}
\begin{Highlighting}[]
\FunctionTok{matrix}\NormalTok{(vector1,}\AttributeTok{nrow=}\DecValTok{3}\NormalTok{, }\AttributeTok{ncol =} \DecValTok{4}\NormalTok{) }
\end{Highlighting}
\end{Shaded}

\begin{verbatim}
##      [,1] [,2] [,3] [,4]
## [1,]    4   18   12   13
## [2,]   14   10    9   17
## [3,]    7    5    1    2
\end{verbatim}

\begin{Shaded}
\begin{Highlighting}[]
\NormalTok{grade\_class}\OtherTok{=}\FunctionTok{runif}\NormalTok{(}\DecValTok{15}\NormalTok{, }\AttributeTok{min=}\DecValTok{1}\NormalTok{, }\AttributeTok{max=}\DecValTok{20}\NormalTok{)}
\NormalTok{grade\_class}\OtherTok{=}\FunctionTok{round}\NormalTok{(grade\_class, }\AttributeTok{digits =} \DecValTok{0}\NormalTok{)}
\NormalTok{grade\_class}\OtherTok{=} \FunctionTok{matrix}\NormalTok{(grade\_class, }\AttributeTok{ncol =} \DecValTok{3}\NormalTok{, }\AttributeTok{nrow =} \DecValTok{5}\NormalTok{)}
\NormalTok{grade\_class}
\end{Highlighting}
\end{Shaded}

\begin{verbatim}
##      [,1] [,2] [,3]
## [1,]    5   20    6
## [2,]   20   10    6
## [3,]   16    9   16
## [4,]   16    9   18
## [5,]   15   12    8
\end{verbatim}

We can also create character type matrices

\begin{Shaded}
\begin{Highlighting}[]
\NormalTok{ma\_matrice2}\OtherTok{=}\FunctionTok{matrix}\NormalTok{(}\FunctionTok{c}\NormalTok{(}\StringTok{"a"}\NormalTok{, }\StringTok{"b"}\NormalTok{, }\StringTok{"c"}\NormalTok{,}\StringTok{"d"}\NormalTok{,}\StringTok{"e"}\NormalTok{,}\StringTok{"f"}\NormalTok{,}\StringTok{"g"}\NormalTok{,}\StringTok{"h"}\NormalTok{,}\StringTok{"i"}\NormalTok{), }\AttributeTok{nrow =} \DecValTok{3}\NormalTok{, }\AttributeTok{ncol =} \DecValTok{3}\NormalTok{)}
\NormalTok{ma\_matrice2}
\end{Highlighting}
\end{Shaded}

\begin{verbatim}
##      [,1] [,2] [,3]
## [1,] "a"  "d"  "g" 
## [2,] "b"  "e"  "h" 
## [3,] "c"  "f"  "i"
\end{verbatim}

In order to name the rows and columns of the matrices

\begin{Shaded}
\begin{Highlighting}[]
\FunctionTok{colnames}\NormalTok{(grade\_class)}\OtherTok{=} \FunctionTok{c}\NormalTok{(}\StringTok{"Biology"}\NormalTok{, }\StringTok{"Maths"}\NormalTok{, }\StringTok{"Physical"}\NormalTok{)}
\FunctionTok{rownames}\NormalTok{(grade\_class)}\OtherTok{=}\FunctionTok{c}\NormalTok{(}\StringTok{"Elise"}\NormalTok{, }\StringTok{"Leon"}\NormalTok{,}\StringTok{"Sully"}\NormalTok{, }\StringTok{"Anouck"}\NormalTok{,}\StringTok{"Baptiste"}\NormalTok{)}
\NormalTok{grade\_class}
\end{Highlighting}
\end{Shaded}

\begin{verbatim}
##          Biology Maths Physical
## Elise          5    20        6
## Leon          20    10        6
## Sully         16     9       16
## Anouck        16     9       18
## Baptiste      15    12        8
\end{verbatim}

\begin{Shaded}
\begin{Highlighting}[]
\NormalTok{grade\_class [, }\StringTok{"Biology"}\NormalTok{] }\DocumentationTok{\#\# displays the first column}
\end{Highlighting}
\end{Shaded}

\begin{verbatim}
##    Elise     Leon    Sully   Anouck Baptiste 
##        5       20       16       16       15
\end{verbatim}

\begin{Shaded}
\begin{Highlighting}[]
\NormalTok{grade\_class [, }\DecValTok{1}\NormalTok{] }\DocumentationTok{\#\# this instruction is equivalent to the previous one}
\end{Highlighting}
\end{Shaded}

\begin{verbatim}
##    Elise     Leon    Sully   Anouck Baptiste 
##        5       20       16       16       15
\end{verbatim}

\begin{Shaded}
\begin{Highlighting}[]
\NormalTok{grade\_class [}\StringTok{"Sully"}\NormalTok{,] }\DocumentationTok{\#\# displays the 3rd line}
\end{Highlighting}
\end{Shaded}

\begin{verbatim}
##  Biology    Maths Physical 
##       16        9       16
\end{verbatim}

\begin{Shaded}
\begin{Highlighting}[]
\NormalTok{grade\_class [}\DecValTok{3}\NormalTok{,] }\DocumentationTok{\#\# instruction equivalent to the previous one}
\end{Highlighting}
\end{Shaded}

\begin{verbatim}
##  Biology    Maths Physical 
##       16        9       16
\end{verbatim}

If we want to see Sully's note in Physical, we have two ways:

\begin{Shaded}
\begin{Highlighting}[]
\NormalTok{grade\_class[}\StringTok{"Sully"}\NormalTok{, }\StringTok{"Physical"}\NormalTok{]}
\end{Highlighting}
\end{Shaded}

\begin{verbatim}
## [1] 16
\end{verbatim}

\begin{Shaded}
\begin{Highlighting}[]
\NormalTok{grade\_class[}\DecValTok{3}\NormalTok{,}\DecValTok{3}\NormalTok{] }\DocumentationTok{\#\#instruction equivalent to the previous one}
\end{Highlighting}
\end{Shaded}

\begin{verbatim}
## [1] 16
\end{verbatim}

\begin{Shaded}
\begin{Highlighting}[]
\NormalTok{grade\_class[}\StringTok{"Elise"}\NormalTok{,}\StringTok{"Maths"}\NormalTok{]}
\end{Highlighting}
\end{Shaded}

\begin{verbatim}
## [1] 20
\end{verbatim}

\begin{Shaded}
\begin{Highlighting}[]
\NormalTok{grade\_class[}\DecValTok{1}\NormalTok{,}\DecValTok{2}\NormalTok{] }\DocumentationTok{\#\# Elise\textquotesingle{}s grade in Maths}
\end{Highlighting}
\end{Shaded}

\begin{verbatim}
## [1] 20
\end{verbatim}

To access Elise's Maths and Physical notes

\begin{Shaded}
\begin{Highlighting}[]
\NormalTok{grade\_class[}\StringTok{"Elise"}\NormalTok{, }\FunctionTok{c}\NormalTok{(}\StringTok{"Maths"}\NormalTok{, }\StringTok{"Physical"}\NormalTok{)]}
\end{Highlighting}
\end{Shaded}

\begin{verbatim}
##    Maths Physical 
##       20        6
\end{verbatim}

\begin{Shaded}
\begin{Highlighting}[]
\NormalTok{grade\_class[}\DecValTok{1}\NormalTok{, }\FunctionTok{c}\NormalTok{(}\DecValTok{2}\NormalTok{,}\DecValTok{3}\NormalTok{)]}
\end{Highlighting}
\end{Shaded}

\begin{verbatim}
##    Maths Physical 
##       20        6
\end{verbatim}

\begin{Shaded}
\begin{Highlighting}[]
\NormalTok{grade\_class[}\StringTok{"Elise"}\NormalTok{, }\DecValTok{2}\SpecialCharTok{:}\DecValTok{3}\NormalTok{]}
\end{Highlighting}
\end{Shaded}

\begin{verbatim}
##    Maths Physical 
##       20        6
\end{verbatim}

We can do the same actions at row level For example see the notes of
Anouck and Baptiste in Biology

\begin{Shaded}
\begin{Highlighting}[]
\NormalTok{grade\_class[}\FunctionTok{c}\NormalTok{(}\StringTok{"Anouck"}\NormalTok{,}\StringTok{"Baptiste"}\NormalTok{), }\StringTok{"Biology"}\NormalTok{]}
\end{Highlighting}
\end{Shaded}

\begin{verbatim}
##   Anouck Baptiste 
##       16       15
\end{verbatim}

\hypertarget{modify-the-elements-of-a-matrix}{%
\subsection{Modify the elements of a
matrix}\label{modify-the-elements-of-a-matrix}}

\begin{Shaded}
\begin{Highlighting}[]
\NormalTok{grade\_class[}\FunctionTok{c}\NormalTok{(}\StringTok{"Anouck"}\NormalTok{,}\StringTok{"Baptiste"}\NormalTok{), }\StringTok{"Biology"}\NormalTok{]}\OtherTok{=} \FunctionTok{c}\NormalTok{(}\DecValTok{15}\NormalTok{, }\DecValTok{17}\NormalTok{)}
\NormalTok{grade\_class[}\FunctionTok{c}\NormalTok{(}\StringTok{"Anouck"}\NormalTok{,}\StringTok{"Baptiste"}\NormalTok{), }\StringTok{"Biology"}\NormalTok{]}
\end{Highlighting}
\end{Shaded}

\begin{verbatim}
##   Anouck Baptiste 
##       15       17
\end{verbatim}

\hypertarget{matrix-operations}{%
\subsection{Matrix operations}\label{matrix-operations}}

In this example, we will use the grade\_class matrix and we will create
a new matrix that we will name matrix\_ope

\begin{Shaded}
\begin{Highlighting}[]
\NormalTok{matrice\_ope }\OtherTok{=} \FunctionTok{c}\NormalTok{(}\FunctionTok{rep}\NormalTok{(}\FloatTok{0.25}\NormalTok{, }\AttributeTok{times =} \DecValTok{5}\NormalTok{), }\FunctionTok{rep}\NormalTok{(}\FloatTok{0.5}\NormalTok{, }\AttributeTok{times =} \DecValTok{5}\NormalTok{), }\FunctionTok{rep}\NormalTok{(}\FloatTok{0.75}\NormalTok{, }\AttributeTok{times =} \DecValTok{5}\NormalTok{)) }\CommentTok{\# the rep function allows you to repeat a number in times = 5 times}
\NormalTok{matrice\_ope}
\end{Highlighting}
\end{Shaded}

\begin{verbatim}
##  [1] 0.25 0.25 0.25 0.25 0.25 0.50 0.50 0.50 0.50 0.50 0.75 0.75 0.75 0.75 0.75
\end{verbatim}

\begin{Shaded}
\begin{Highlighting}[]
\NormalTok{matrice\_ope}\OtherTok{=}\FunctionTok{matrix}\NormalTok{(matrice\_ope, }\AttributeTok{nrow =} \DecValTok{5}\NormalTok{, }\AttributeTok{ncol =} \DecValTok{3}\NormalTok{) }\CommentTok{\# transformation of the vector into a matrix.}
\NormalTok{matrice\_ope}
\end{Highlighting}
\end{Shaded}

\begin{verbatim}
##      [,1] [,2] [,3]
## [1,] 0.25  0.5 0.75
## [2,] 0.25  0.5 0.75
## [3,] 0.25  0.5 0.75
## [4,] 0.25  0.5 0.75
## [5,] 0.25  0.5 0.75
\end{verbatim}

\hypertarget{addition-of-two-matrices}{%
\subsection{Addition of two matrices}\label{addition-of-two-matrices}}

\begin{Shaded}
\begin{Highlighting}[]
\NormalTok{grade\_class}
\end{Highlighting}
\end{Shaded}

\begin{verbatim}
##          Biology Maths Physical
## Elise          5    20        6
## Leon          20    10        6
## Sully         16     9       16
## Anouck        15     9       18
## Baptiste      17    12        8
\end{verbatim}

\begin{Shaded}
\begin{Highlighting}[]
\NormalTok{grade\_class }\SpecialCharTok{+}\NormalTok{ matrice\_ope}
\end{Highlighting}
\end{Shaded}

\begin{verbatim}
##          Biology Maths Physical
## Elise       5.25  20.5     6.75
## Leon       20.25  10.5     6.75
## Sully      16.25   9.5    16.75
## Anouck     15.25   9.5    18.75
## Baptiste   17.25  12.5     8.75
\end{verbatim}

\hypertarget{multiplication-of-two-matrices}{%
\subsection{Multiplication of two
matrices}\label{multiplication-of-two-matrices}}

In R it is not a matrix calculation where it is a question of
multiplying the first row of the first matrix by the first column etc.
It is very basic, first column of the first matrix multiplied by first
column of the second matrix.

\begin{Shaded}
\begin{Highlighting}[]
\NormalTok{grade\_class}
\end{Highlighting}
\end{Shaded}

\begin{verbatim}
##          Biology Maths Physical
## Elise          5    20        6
## Leon          20    10        6
## Sully         16     9       16
## Anouck        15     9       18
## Baptiste      17    12        8
\end{verbatim}

\begin{Shaded}
\begin{Highlighting}[]
\NormalTok{matrice\_ope}
\end{Highlighting}
\end{Shaded}

\begin{verbatim}
##      [,1] [,2] [,3]
## [1,] 0.25  0.5 0.75
## [2,] 0.25  0.5 0.75
## [3,] 0.25  0.5 0.75
## [4,] 0.25  0.5 0.75
## [5,] 0.25  0.5 0.75
\end{verbatim}

\begin{Shaded}
\begin{Highlighting}[]
\NormalTok{grade\_class }\SpecialCharTok{*}\NormalTok{ matrice\_ope}
\end{Highlighting}
\end{Shaded}

\begin{verbatim}
##          Biology Maths Physical
## Elise       1.25  10.0      4.5
## Leon        5.00   5.0      4.5
## Sully       4.00   4.5     12.0
## Anouck      3.75   4.5     13.5
## Baptiste    4.25   6.0      6.0
\end{verbatim}

Likewise for the division

\begin{Shaded}
\begin{Highlighting}[]
\NormalTok{grade\_class}
\end{Highlighting}
\end{Shaded}

\begin{verbatim}
##          Biology Maths Physical
## Elise          5    20        6
## Leon          20    10        6
## Sully         16     9       16
## Anouck        15     9       18
## Baptiste      17    12        8
\end{verbatim}

\begin{Shaded}
\begin{Highlighting}[]
\NormalTok{matrice\_ope}
\end{Highlighting}
\end{Shaded}

\begin{verbatim}
##      [,1] [,2] [,3]
## [1,] 0.25  0.5 0.75
## [2,] 0.25  0.5 0.75
## [3,] 0.25  0.5 0.75
## [4,] 0.25  0.5 0.75
## [5,] 0.25  0.5 0.75
\end{verbatim}

\begin{Shaded}
\begin{Highlighting}[]
\NormalTok{grade\_class}\SpecialCharTok{/}\NormalTok{matrice\_ope}
\end{Highlighting}
\end{Shaded}

\begin{verbatim}
##          Biology Maths Physical
## Elise         20    40  8.00000
## Leon          80    20  8.00000
## Sully         64    18 21.33333
## Anouck        60    18 24.00000
## Baptiste      68    24 10.66667
\end{verbatim}

\hypertarget{the-subtraction-of-two-matrices}{%
\subsection{The subtraction of two
matrices}\label{the-subtraction-of-two-matrices}}

\begin{Shaded}
\begin{Highlighting}[]
\NormalTok{grade\_class}
\end{Highlighting}
\end{Shaded}

\begin{verbatim}
##          Biology Maths Physical
## Elise          5    20        6
## Leon          20    10        6
## Sully         16     9       16
## Anouck        15     9       18
## Baptiste      17    12        8
\end{verbatim}

\begin{Shaded}
\begin{Highlighting}[]
\NormalTok{matrice\_ope}
\end{Highlighting}
\end{Shaded}

\begin{verbatim}
##      [,1] [,2] [,3]
## [1,] 0.25  0.5 0.75
## [2,] 0.25  0.5 0.75
## [3,] 0.25  0.5 0.75
## [4,] 0.25  0.5 0.75
## [5,] 0.25  0.5 0.75
\end{verbatim}

\begin{Shaded}
\begin{Highlighting}[]
\NormalTok{grade\_class}\SpecialCharTok{{-}}\NormalTok{matrice\_ope}
\end{Highlighting}
\end{Shaded}

\begin{verbatim}
##          Biology Maths Physical
## Elise       4.75  19.5     5.25
## Leon       19.75   9.5     5.25
## Sully      15.75   8.5    15.25
## Anouck     14.75   8.5    17.25
## Baptiste   16.75  11.5     7.25
\end{verbatim}

\hypertarget{some-additional-functions-on-matrices}{%
\subsection{Some additional functions on
matrices}\label{some-additional-functions-on-matrices}}

\begin{Shaded}
\begin{Highlighting}[]
\FunctionTok{colSums}\NormalTok{(grade\_class)}
\end{Highlighting}
\end{Shaded}

\begin{verbatim}
##  Biology    Maths Physical 
##       73       60       54
\end{verbatim}

\begin{Shaded}
\begin{Highlighting}[]
\FunctionTok{rowSums}\NormalTok{(grade\_class)}
\end{Highlighting}
\end{Shaded}

\begin{verbatim}
##    Elise     Leon    Sully   Anouck Baptiste 
##       31       36       41       42       37
\end{verbatim}

\begin{Shaded}
\begin{Highlighting}[]
\FunctionTok{colMeans}\NormalTok{(grade\_class)}
\end{Highlighting}
\end{Shaded}

\begin{verbatim}
##  Biology    Maths Physical 
##     14.6     12.0     10.8
\end{verbatim}

\begin{Shaded}
\begin{Highlighting}[]
\FunctionTok{round}\NormalTok{(}\FunctionTok{rowMeans}\NormalTok{(grade\_class),}\AttributeTok{digits =} \DecValTok{2}\NormalTok{)}
\end{Highlighting}
\end{Shaded}

\begin{verbatim}
##    Elise     Leon    Sully   Anouck Baptiste 
##    10.33    12.00    13.67    14.00    12.33
\end{verbatim}

\end{document}
